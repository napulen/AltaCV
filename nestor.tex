%%%%%%%%%%%%%%%%%
% This is an sample CV template created using altacv.cls
% (v1.6.2, 28 Aug 2021) written by LianTze Lim (liantze@gmail.com). Now compiles with pdfLaTeX, XeLaTeX and LuaLaTeX.
%
%% It may be distributed and/or modified under the
%% conditions of the LaTeX Project Public License, either version 1.3
%% of this license or (at your option) any later version.
%% The latest version of this license is in
%%    http://www.latex-project.org/lppl.txt
%% and version 1.3 or later is part of all distributions of LaTeX
%% version 2003/12/01 or later.
%%%%%%%%%%%%%%%%

%% Use the "normalphoto" option if you want a normal photo instead of cropped to a circle
% \documentclass[10pt,a4paper,normalphoto]{altacv}

\documentclass[10pt,a4paper,ragged2e,withhyper]{altacv}
%% AltaCV uses the fontawesome5 and packages.
%% See http://texdoc.net/pkg/fontawesome5 for full list of symbols.

% Change the page layout if you need to
\geometry{left=1.25cm,right=1.25cm,top=1.5cm,bottom=1.5cm,columnsep=1.2cm}

% The paracol package lets you typeset columns of text in parallel
\usepackage{paracol}

% Change the font if you want to, depending on whether
% you're using pdflatex or xelatex/lualatex
\ifxetexorluatex
  % If using xelatex or lualatex:
  \setmainfont{Roboto Slab}
  \setsansfont{Lato}
  \renewcommand{\familydefault}{\sfdefault}
\else
  % If using pdflatex:
  \usepackage[rm]{roboto}
  \usepackage[defaultsans]{lato}
  % \usepackage{sourcesanspro}
  \renewcommand{\familydefault}{\sfdefault}
\fi

% Change the colours if you want to
\definecolor{SlateGrey}{HTML}{2E2E2E}
\definecolor{LightGrey}{HTML}{666666}
\definecolor{DarkPastelRed}{HTML}{450808}
\definecolor{PastelRed}{HTML}{8F0D0D}
\definecolor{GoldenEarth}{HTML}{E7D192}
\definecolor{VividPurple}{HTML}{3E0097}
\definecolor{DarkBlue}{HTML}{000044}
\definecolor{MiddleBlue}{HTML}{1B4F72}
\definecolor{LightBlue}{HTML}{2E86C1}
\definecolor{DarkBrown}{HTML}{3F2A14}

\colorlet{name}{SlateGrey}
\colorlet{tagline}{LightGrey}
\colorlet{heading}{LightBlue}
\colorlet{headingrule}{LightBlue}
% \colorlet{subheading}{LightBlue}
\colorlet{accent}{MiddleBlue}
\colorlet{emphasis}{SlateGrey}
\colorlet{body}{LightGrey}

% Change some fonts, if necessary
% \renewcommand{\namefont}{\Huge\rmfamily\bfseries}
% \renewcommand{\personalinfofont}{\footnotesize}
% \renewcommand{\cvsectionfont}{\LARGE\rmfamily\bfseries}
% \renewcommand{\cvsubsectionfont}{\large\bfseries}


% Change the bullets for itemize and rating marker
% for \cvskill if you want to
\renewcommand{\itemmarker}{{\small\textbullet}}
\renewcommand{\ratingmarker}{\faCircle}

%% Use (and optionally edit if necessary) this .cfg if you
%% want to use an author-year reference style like APA(6)
%% for your publication list
% % When using APA6 if you need more author names to be listed
% because you're e.g. the 12th author, add apamaxprtauth=12
\usepackage[backend=biber,style=apa6,sorting=ydnt]{biblatex}
\defbibheading{pubtype}{\cvsubsection{#1}}
\renewcommand{\bibsetup}{\vspace*{-\baselineskip}}
\AtEveryBibitem{\makebox[\bibhang][l]{\itemmarker}}
\setlength{\bibitemsep}{0.25\baselineskip}
\setlength{\bibhang}{1.25em}


%% Use (and optionally edit if necessary) this .cfg if you
%% want an originally numerical reference style like IEEE
%% for your publication list
\usepackage[backend=biber,style=ieee,sorting=ydnt]{biblatex}
%% For removing numbering entirely when using a numeric style
\setlength{\bibhang}{1.25em}
\DeclareFieldFormat{labelnumberwidth}{\makebox[\bibhang][l]{\itemmarker}}
\setlength{\biblabelsep}{0pt}
\defbibheading{pubtype}{\cvsubsection{#1}}
\renewcommand{\bibsetup}{\vspace*{-\baselineskip}}


%% sample.bib contains your publications
\addbibresource{nestor.bib}

\begin{document}
\name{N\'estor N\'apoles L\'opez}
\tagline{PhD Candidate in Music Technology}
%% You can add multiple photos on the left or right
% \photoR{2.8cm}{Globe_High}
% \photoL{2.5cm}{Yacht_High,Suitcase_High}

\personalinfo{%
  % Not all of these are required!
  \homepage{napulen.github.io}
  \email{napulen@gmail.com}
  \twitter{@napulen}
  \github{napulen}
%   \linkedin{your_id}
  \location{Montr\'eal, QC}
  \phone{+1 438-979-1473}
%   \mailaddress{Åddrésş, Street, 00000 Cóuntry}
%   \orcid{0000-0000-0000-0000}
  %% You can add your own arbitrary detail with
  %% \printinfo{symbol}{detail}[optional hyperlink prefix]
  % \printinfo{\faPaw}{Hey ho!}[https://example.com/]
  %% Or you can declare your own field with
  %% \NewInfoFiled{fieldname}{symbol}[optional hyperlink prefix] and use it:
  % \NewInfoField{gitlab}{\faGitlab}[https://gitlab.com/]
  % \gitlab{your_id}
  %%
  %% For services and platforms like Mastodon where there isn't a
  %% straightforward relation between the user ID/nickname and the hyperlink,
  %% you can use \printinfo directly e.g.
  % \printinfo{\faMastodon}{@username@instace}[https://instance.url/@username]
  %% But if you absolutely want to create new dedicated info fields for
  %% such platforms, then use \NewInfoField* with a star:
  % \NewInfoField*{mastodon}{\faMastodon}
  %% then you can use \mastodon, with TWO arguments where the 2nd argument is
  %% the full hyperlink.
  % \mastodon{@username@instance}{https://instance.url/@username}
}

\makecvheader
%% Depending on your tastes, you may want to make fonts of itemize environments slightly smaller
% \AtBeginEnvironment{itemize}{\small}

%% Set the left/right column width ratio to 6:4.
\columnratio{0.6}

% Start a 2-column paracol. Both the left and right columns will automatically
% break across pages if things get too long.
\begin{paracol}{2}

\cvsection{About me}
\begin{quote}
I am a computational music theorist with a strong passion for tonal music.
As a young person, I yearned to access the basics of music theory.
Nowadays, I spend my time designing technology that will democratize this access for others.
\end{quote}

\cvsection{Experience}

\cvevent{PhD Casual Research Assistant}{Distributed Digital Music Archives and Libraries Lab}{September 2017 -- Ongoing}{Montr\'eal, Qu\'ebec}
My research focuses on:
\begin{itemize}
    \item The computational modeling~\parencite{napoles_lopez_augmentednet_2021} and digital annotation of Roman numeral analyses~\parencite{napoles_lopez_harmalysis_2020}, \textbf{which is the topic of my PhD dissertation}
    \item Automatic key estimation \cite{napoles_lopez_key-finding_2019, napoles_lopez_local_2020}
    \item Encoding and evaluating symbolic music files \parencite{degroot-maggetti_data_2020, napoles_lopez_harmonic_2020, napoles_lopez_encoding_2018, napoles_lopez_effects_2019}
    \item Gamification of musicianship and music theory exercises \parencite{napoles_lopez_-re-myth_2020, napoles_lopez_dandelot_2019}
\end{itemize}

Additionally, as a research assistant for DDMAL, I maintain the django-based \emph{Cantus Ultimus} website, train new hires regarding good version control practices, and support the infrastructure of the lab in the ``Arbutus'' cloud, which is part of Compute Canada.

\medskip

\cvevent{MSc Research Assistant}{CompMusic}{November 2016 -- June 2017}{Barcelona, Spain}
I participated in a project called \emph{CompMusic}.
There, I annotated a dataset of Classical string quartets with \textasciitilde{}5,000 Roman numeral labels. 
I used this data to evaluate \emph{Melisma}, a complex algorithm for music analysis developed by Temperley and Sleator.
The work in this research assistantship became the bulk of my Masters thesis.

\medskip

\cvevent{Graphics Software Security Engineer}{Intel Corporation}{October 2014 -- August 2016}{Guadalajara, M\'exico}
Tool development and software security analysis of 3D-graphics libraries. 
% I found and disclosed \textasciitilde{}5 medium-and-high security vulnerabilities in graphic libraries through fuzzing and API-library injection. 
In 2015, I co-authored a conference presentation about API-Injection techniques for fuzzing graphics DLLs, in Portland, OR.

\medskip

\cvevent{Lead Software Developer}{CINVESTAV / Modutram}{January 2014 -- September 2014}{Guadalajara, M\'exico}
C++ developer for a public-transportation startup. 
I was the main developer of a traffic simulation suite written in C++, Qt, and OpenGL.

\medskip

\cvevent{Software Validation Engineer}{CINVESTAV / Intel Corporation}{August 2013 -- January 2014}{Guadalajara, M\'exico}
Writing scripts and software tools (mostly in Python and Bash) for the automatic validation of software and hardware components.

% use ONLY \newpage if you want to force a page break for
% ONLY the current column
% \newpage

\cvsection{Publications}

\nocite{*}


\printbibliography[heading=pubtype,title={\printinfo{\faBookOpen}{Peer-Reviewed Conference with Proceedings}},keyword=mypublications\_proceedings]

% \divider

\printbibliography[heading=pubtype,title={\printinfo{\faUsers}{Peer-Reviewed Conference without Proceedings}},keyword=mypublications\_noproceedings]

% \divider

\printbibliography[heading=pubtype,title={\printinfo{\faUserGraduate}{Thesis}},keyword=mypublications\_thesis]

%% Switch to the right column. This will now automatically move to the second
%% page if the content is too long.
\switchcolumn

\cvsection{Education}
{\faGraduationCap} \cvevent{PhD in Music Technology}{McGill University. Montr\'eal, Qu\'ebec}{September 2017 -- Ongoing}{}
Provisional thesis title: \emph{Multitask tonal analysis in symbolic music representations}

\medskip

{\faGraduationCap} \cvevent{MSc in Sound and Music Computing}{Universitat Pompeu Fabra. Barcelona, Spain}{September 2016 -- September 2017}{}
Thesis: \emph{Automatic Harmonic Analysis of Classical String Quartets from Symbolic Score}~\parencite{napoles_lopez_automatic_2017}.

\medskip

{\faGraduationCap} \cvevent{Medium-Professional Degree in Classical Piano Performance}{National Institute of Fine Arts. Guadalajara, M\'exico}{August 2009 -- July 2013}{}
\medskip

{\faGraduationCap} \cvevent{Licenciatura en Inform\'atica}{University of Guadalajara. M\'exico}{August 2008 -- January 2013}{}

\cvsection{Skills}
\cvskill{Python}{5}
\cvskill{C / C++}{4.5}
\cvskill{Bash}{4}
\cvskill{HTML / CSS / JavaScript}{3.5}
\cvskill{x86-64 Assembly}{2.5}
\divider
\cvskill{Tensorflow / Keras}{5}
\cvskill{pandas, numpy, seaborn}{4.5}
\cvskill{pytorch, sklearn}{3}
\divider
\cvskill{git, version control}{5}
\cvskill{django}{4.5}
\cvskill{gh-pages, gh-actions}{4}
\cvskill{unittest, coverage}{4}
\cvskill{devops}{3.5}
\divider
\cvskill{music21, mido}{5}
\cvskill{MusicXML / MIDI}{5}
\cvskill{Audio Signal Processing}{3}
\divider
\cvskill{Delivering presentations}{5}
\cvskill{Collaboration}{4.5}
\cvskill{Teaching}{4}
\cvskill{People management}{2.5}


\cvsection{Interests}

% Adapted from @Jake's answer from http://tex.stackexchange.com/a/82729/226
% \wheelchart{outer radius}{inner radius}{
% comma-separated list of value/text width/color/detail}
\wheelchart{0.5cm}{1cm}{%
  9/4em/accent!80/Research,
  8/4em/accent!70/Music theory,
  7/4em/accent!60/Machine learning,
  6/4em/accent!50/Coding,
  3/8em/accent!20/Game development
}

\cvsection{Teaching}

{\faVolumeUp} \cvevent{MUMT 301}{McGill University. Montr\'eal, Qu\'ebec}{Fall 2020, Fall 2021}{}
Co-lecturer of the course ``MUMT 301: Music and the Internet'', which is taught to undergrad students of the Musical Applications of Technology (MAT) and Musical Science & Technology (MST) minors at McGill.

The course briefly touches on HTML, CSS, and JavaScript development, the history of the Internet (from the creation of ARPA, to the invention of the Web browser), network protocols, JavaScript-based sound generation, and digital symbolic-and-audio formats for music representation.

\medskip

{\faHandsHelping} \cvevent{WiMIR Mentor}{Women in Music Information Retrieval (WiMIR)}{2019, 2020}{}

I have volunteered twice as a mentor for the WiMIR initiative.
Through this project, I have met two amazing students who are interested in the Music Information Retrieval (MIR) field.
In both occasions, we set biweekly meetings to discuss academic life, papers, and possible research directions they might pursue. 

\cvsection{Languages}
\cvskill{Spanish}{5}
\cvskill{English}{4.5}
\cvskill{French}{2.5}
\cvskill{Polish}{1}


\cvsection{Awards}
\cvachievement{\faUniversity}{Doctoral Research Scholarship}{Fonds de recherche du Québec--Société et culture (FRQSC), 2019-2021}
\cvachievement{\faGamepad}{Student Project Award: Ear Training Gamification Software}{Centre for Interdisciplinary Research in Music Media and Technology (CIRMMT), 2019}
\cvachievement{\faMoneyBill}{\emph{INPhINIT} fellowship (declined in favor of McGill's scholarship)}{La Caixa Foundation, 2017}
\cvachievement{\faLaptopCode}{Best workshop at the Software Professionals Conference (SWPC)}{Intel Corporation, August 2015}

% \cvsection{Languages}

% \cvskill{English}{5}
% \divider

% \cvskill{Spanish}{4}
% \divider

% \cvskill{German}{3.5} %% Supports X.5 values.

%% Yeah I didn't spend too much time making all the
%% spacing consistent... sorry. Use \smallskip, \medskip,
%% \bigskip, \vspace etc to make adjustments.
% \medskip

% \cvsection{Referees}

% % \cvref{name}{email}{mailing address}
% \cvref{Prof.\ Alpha Beta}{Institute}{a.beta@university.edu}
% {Address Line 1\\Address line 2}

% \divider

% \cvref{Prof.\ Gamma Delta}{Institute}{g.delta@university.edu}
% {Address Line 1\\Address line 2}

\end{paracol}


\end{document}
